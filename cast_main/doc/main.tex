\documentclass[a4paper]{article}

\usepackage{geometry}
\geometry{left=2.5cm,right=2.5cm,top=2.5cm,bottom=2.5cm}

% Basics
\usepackage{fixltx2e}
\usepackage{url}
\usepackage{fancyvrb}
\usepackage{mdwlist}  % Miscellaneous list-related commands
\usepackage{xspace}   % Smart spacing
\usepackage{supertabular}

% https://www.nesono.com/?q=book/export/html/347
% Package for inserting TODO statements in nice colorful boxes - so that you
% won’t forget to fix/remove them. To add a todo statement, use something like
% \todo{Find better wording here}.
\usepackage{todonotes}

%% Math
\usepackage{amsmath}
\usepackage{amsthm}
\usepackage{amssymb}
\usepackage{bm}       % Bold symbols in maths mode
\usepackage{wasysym}  % LHD and RHD

% http://tex.stackexchange.com/questions/114151/how-do-i-reference-in-appendix-a-theorem-given-in-the-body
\usepackage{thmtools, thm-restate}

%% Theoretical computer science
\usepackage{stmaryrd}
\usepackage{mathtools}  % For "::=" ( \Coloneqq )

%% Code listings
\usepackage{listings}

%% Font
\usepackage[euler-digits,euler-hat-accent]{eulervm}

% circle (use tikz package)
\newcommand{\ascriptionnode}[1]{
  \tikz{
  \node[shape=circle,draw=black,inner sep=1.5pt,scale=0.3] at (2em, 2em) {#1}; }}


\newcommand{\ascriptioncircle}[2][black,fill=black]{\tikz[baseline=-0.5ex]\draw[#1,radius=#2] (0,0) circle;}

\newcommand{\whitecircle}{\ascriptioncircle[black,fill=white]{2pt}}
\newcommand{\blackcircle}{\ascriptioncircle{2pt}}
\newcommand*\circled[1]{\tikz[baseline=(char.base)]{
    \node[shape=circle,fill,inner sep=1pt] (char) {\textcolor{white}{#1}};}}

%% Ott
\usepackage{ottalt}
\newcommand{\hlmath}[2][gray!40]{%
	\colorbox{#1}{$\displaystyle#2$}}

% Ott includes
\inputott{rules}


\title{Iso $\mathsf{\lambda}_{\mu}$ with casting}
\author{Tony}

\begin{document}

\maketitle


% \begin{align*}
%   &\text{Type} &A, B&::= [[X]] ~|~ [[Int]] ~|~ [[A-> B]] ~|~ [[A&B]] ~|~ [[Top]] ~|~ [[Bot]]  ~|~ [[Forall X*A.B]] ~|~ [[{l:A}]] \\
%   &\text{Expressions} &e    &::= x ~|~ i ~|~ T ~|~ e:A ~|~ [[e1 e2]] ~|~ [[ \ x : A .e]]~|~ [[e_1,,e_2]]~|~[[fix x : A . e]] \\
%   &&& ~|~ [[{l=e}]] ~|~ [[e.l]] ~|~ [[ \X.e]] ~|~ [[ e A ]]\\
%   &\text{Values} &v   &::= i ~|~[[T]]~|~ [[ (\ x : A .e) : B]] ~|~ [[\ x : A .e]] ~|~ [[v_1,,v_2]]~|~ [[{l=e}:A]] ~|~ [[{l=e}]] \\
%   &&& ~|~ [[ (\X.e):B ]] ~|~ [[ \X.e ]] \\
%   &\text{Pre-values} &[[u]] &::= i ~|~ [[T]] ~|~[[e:A]]~|~ [[u_1,,u_2]]\\
%   &\text{Term contexts} &[[G]] &::=  [[ [] ]] ~|~ [[ G, x:A ]] \\
%   &\text{Type contexts} &[[D]] &::=  [[ [] ]] ~|~ [[ D, X*A ]]\\
%   &\text{Arguments} &[[arg]] &::= [[e]] ~|~ [[{l}]] ~|~ [[A]]\\
% \end{align*}

\section{Grammar}

\ottgrammar

\section{Subtyping}

\ottdefnsWellFormedTypeEnv


\section{Typing}

\ottdefnsWellFormedTermEnv

\ottdefnsWellFormedType

\ottdefnsTypCast

\ottdefnsTyping



\section{Semantics}

\ottdefnsValues

% \ottdefnsDualCast

\ottdefnsReduction


\end{document}